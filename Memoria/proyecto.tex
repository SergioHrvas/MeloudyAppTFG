\documentclass[a4paper,11pt]{book}
%\documentclass[a4paper,twoside,11pt,titlepage]{book}
\usepackage{listings}
\usepackage[utf8]{inputenc}
\usepackage[spanish]{babel}
\usepackage{float}
\usepackage{subfig}% \usepackage[style=list, number=none]{glossary} %
%\usepackage{titlesec}
%\usepackage{pailatino}
\usepackage[font=scriptsize]{caption}
\decimalpoint
\usepackage{dcolumn}
\newcolumntype{.}{D{.}{\esperiod}{-1}}
\makeatletter
\addto\shorthandsspanish{\let\esperiod\es@period@code}
\makeatother
% \usepackage{colortbl}
\usepackage[xcdraw]{xcolor}
%\usepackage[chapter]{algorithm}
\RequirePackage{verbatim}
%\RequirePackage[Glenn]{fncychap}
\usepackage{fancyhdr}
\usepackage{graphicx}
\usepackage[export]{adjustbox}
\usepackage{afterpage}
\usepackage{makecell}
\usepackage{longtable}
\usepackage{tabularx}
\usepackage{hhline}
\usepackage[pdfborder={0 0 0}]{hyperref} %referencia
\usepackage{cite}

% ********************************************************************
% Re-usable information
% ********************************************************************
% \newcommand{\myTitle}{Aplicación multiplataforma para el aprendizaje del lenguaje musical\xspace}
% \newcommand{\myDegree}{Grado en Ingeniería Informática\xspace}
% \newcommand{\myName}{Sergio Hervás Cobo (alumno)\xspace}
% \newcommand{\myProf}{Luis López Escudero (tutor1)\xspace}
% \newcommand{\myOtherProf}{Germán Arroyo Moreno (tutor2)\xspace}
% %\newcommand{\mySupervisor}{Put name here\xspace}
% \newcommand{\myFaculty}{Escuela Técnica Superior de Ingenierías Informática y de
% Telecomunicación\xspace}
% \newcommand{\myFacultyShort}{E.T.S. de Ingenierías Informática y de
% Telecomunicación\xspace}
% \newcommand{\myDepartment}{Departamento de ...\xspace}
% \newcommand{\myUni}{\protect{Universidad de Granada}\xspace}
% \newcommand{\myLocation}{Granada\xspace}
% \newcommand{\myTime}{\today\xspace}
% \newcommand{\myVersion}{Version 0.1\xspace}

% \hypersetup{
% pdfauthor = {\myName (email (en) ugr (punto) es)},
% pdftitle = {\myTitle},
% pdfsubject = {},
% pdfkeywords = {palabra_clave1, palabra_clave2, palabra_clave3, ...},
% pdfcreator = {LaTeX con el paquete ....},
% pdfproducer = {pdflatex}
% }

%\hyphenation{}


%\usepackage{doxygen/doxygen}
%\usepackage{pdfpages}
\usepackage{url}
\usepackage{colortbl,longtable}
\usepackage[stable]{footmisc}
\usepackage{wrapfig}
%\usepackage{index}

%\makeindex
%\usepackage[style=long, cols=2,border=plain,toc=true,number=none]{glossary}
% \makeglossary

% Definición de comandos que me son tiles:
%\renewcommand{\indexname}{Índice alfabético}
%\renewcommand{\glossaryname}{Glosario}

\pagestyle{fancy}
\fancyhf{}
\fancyhead[LO]{\leftmark}
\fancyhead[RE]{\rightmark}
\fancyhead[RO,LE]{\textbf{\thepage}}
\renewcommand{\chaptermark}[1]{\markboth{\textbf{#1}}{}}
\renewcommand{\sectionmark}[1]{\markright{\textbf{\thesection. #1}}}
\setcounter{secnumdepth}{5} 
\setlength{\headheight}{1.5\headheight}

\newcommand{\HRule}{\rule{\linewidth}{0.5mm}}
%Definimos los tipos teorema, ejemplo y definición podremos usar estos tipos
%simplemente poniendo \begin{teorema} \end{teorema} ...
\newtheorem{teorema}{Teorema}[chapter]
\newtheorem{ejemplo}{Ejemplo}[chapter]
\newtheorem{definicion}{Definición}[chapter]

\definecolor{gray97}{gray}{.97}
\definecolor{gray75}{gray}{.75}
\definecolor{gray45}{gray}{.45}
\definecolor{gray30}{gray}{.94}

\lstset{ frame=Ltb,
     framerule=0.5pt,
     aboveskip=0.5cm,
     framextopmargin=3pt,
     framexbottommargin=3pt,
     framexleftmargin=0.1cm,
     framesep=0pt,
     rulesep=.4pt,
     backgroundcolor=\color{gray97},
     rulesepcolor=\color{black},
     %
     stringstyle=\ttfamily,
     showstringspaces = false,
     basicstyle=\scriptsize\ttfamily,
     commentstyle=\color{gray45},
     keywordstyle=\bfseries,
     %
     numbers=left,
     numbersep=6pt,
     numberstyle=\tiny,
     numberfirstline = false,
     breaklines=true,
   }
 
% minimizar fragmentado de listados
\lstnewenvironment{listing}[1][]
   {\lstset{#1}\pagebreak[0]}{\pagebreak[0]}

\lstdefinestyle{CodigoC}
   {
	basicstyle=\scriptsize,
	frame=single,
	language=C,
	numbers=left
   }
\lstdefinestyle{CodigoC++}
   {
	basicstyle=\small,
	frame=single,
	backgroundcolor=\color{gray30},
	language=C++,
	numbers=left
   }

 
\lstdefinestyle{Consola}
   {basicstyle=\scriptsize\bf\ttfamily,
    backgroundcolor=\color{gray30},
    frame=single,
    numbers=none
   }


\newcommand{\bigrule}{\titlerule[0.5mm]}


%Para conseguir que en las páginas en blanco no ponga cabecerass
\makeatletter
\def\clearpage{%
  \ifvmode
    \ifnum \@dbltopnum =\m@ne
      \ifdim \pagetotal <\topskip
        \hbox{}
      \fi
    \fi
  \fi
  \newpage
  \thispagestyle{empty}
  \write\m@ne{}
  \vbox{}
  \penalty -\@Mi
}
\makeatother

\usepackage{pdfpages}



\begin{document}
   % Plantilla portada UGR
   \input{portada/portada}

   % Plantilla prefacio UGR
   \chapter*{}


\thispagestyle{empty}

\begin{center}
{\large\bfseries Aplicación multiplataforma para el aprendizaje del lenguaje musical: Meloudy}\\
\end{center}
\begin{center}
Sergio Hervás Cobo\\
\end{center}

%\vspace{0.7cm}
\noindent{\textbf{Palabras clave}: Aplicacion, Multiplataforma, Aprendizaje, Lenguaje Musical, Musica, Gamificación}\\

\vspace{0.7cm}
\noindent{\textbf{Resumen}}\\
En los últimos años, debido a las necesidades provocadas por la COVID-19, ha surgido un gran número de aplicaciones móviles dedicadas al aprendizaje y la docencia por gamificación en distintos campos: matemáticas, idiomas, historia, arte... Sin embargo, no existen aplicaciones dedicadas al aprendizaje del lenguaje musical con un enfoque lúdico e interactivo.\\

La falta de aplicaciones de esta temática y el objetivo de fomentar el aprendizaje del lenguaje musical alejado de los métodos tradicionales, han sido los motivos principales para el desarrollo de este proyecto.\\

Se ha desarrollado una aplicación multiplataforma para el aprendizaje del lenguaje musical, que permite a los usuarios aprender de forma interactiva y amena los conceptos básicos de la teoría musical con preguntas variadas y un sistema de logros para favorecer la motivación.

\cleardoublepage


\thispagestyle{empty}


\begin{center}
{\large\bfseries Multiplatform application for learning musical language: Meloudy }\\
\end{center}
\begin{center}
Sergio Hervás Cobo\\
\end{center}

%\vspace{0.7cm}
\noindent{\textbf{Keywords}: Application, Multiplatform, Learning, Musical Language, Music, Gamification}\\

\vspace{0.7cm}
\noindent{\textbf{Abstract}}\\

In recent years, due to the needs caused by COVID-19, a large number of mobile applications have emerged dedicated to gamification learning and teaching in different fields: mathematics, languages, history, art... 
However, there are no applications dedicated to learning musical language with a playful and interactive approach.\\

The lack of applications of this theme and the objective of promoting the learning of musical language away from traditional methods, have been the main reasons for the development of this project.\\

A multi-platform application for learning musical language has been developed, which allows users to learn the basic concepts of music theory in an interactive and entertaining way with varied questions and an achievement system to promote motivation.



\chapter*{}
\thispagestyle{empty}

\noindent\rule[-1ex]{\textwidth}{2pt}\\[4.5ex]

Yo, \textbf{Sergio Hervás Cobo}, alumno de la titulación GRADO EN INGENIERÍA INFORMÁTICA de la \textbf{Escuela Técnica Superior
de Ingenierías Informática y de Telecomunicación de la Universidad de Granada}, con DNI 77023574R, autorizo la
ubicación de la siguiente copia de mi Trabajo Fin de Grado en la biblioteca del centro para que pueda ser
consultada por las personas que lo deseen.

\vspace{6cm}

\noindent Fdo: Sergio Hervás Cobo

\vspace{2cm}

\begin{flushright}
Granada a 12 de Julio de 2023.
\end{flushright}


\chapter*{}
\thispagestyle{empty}

\noindent\rule[-1ex]{\textwidth}{2pt}\\[4.5ex]

D. \textbf{Luis López Escudero}, Profesor del Departamento de Lenguajes y Sistemas Informáticos de la Universidad de Granada.

\vspace{0.5cm}

D. \textbf{Germán Arroyo Moreno}, Profesor del Departamento de Lenguajes y Sistemas Informáticos de la Universidad de Granada.


\vspace{0.5cm}

\textbf{Informan:}

\vspace{0.5cm}

Que el presente trabajo, titulado \textit{\textbf{Aplicación multiplataforma para el aprendizaje del lenguaje musical. Meloudy}},
ha sido realizado bajo su supervisión por \textbf{Sergio Hervás Cobo}, y autorizamos la defensa de dicho trabajo ante el tribunal
que corresponda.

\vspace{0.5cm}

Y para que conste, expiden y firman el presente informe en Granada a 12 de Julio de 2023 .

\vspace{1cm}

\textbf{Los directores:}

\vspace{5cm}

\noindent \textbf{Luis López Escudero \ \ \ \ \ Germán Arroyo Moreno}

\chapter*{Agradecimientos}
\thispagestyle{empty}

       \vspace{1cm}

A mi madre, por su apoyo incondicional y por creer siempre en mí. 

A mi padre, porque aunque no esté aquí sé que estaría orgulloso de lo que he conseguido. 

A Migue, mi pareja, por su paciencia durante los años de carrera y darme ánimos cuando más lo necesitaba. 

A Luis López, mi tutor, por su ayuda y consejos durante el desarrollo de este proyecto. 
   \frontmatter

   % Índice de contenidos
   \newpage
   \tableofcontents

   % Índice de imágenes y tablas 
   \newpage
   \listoffigures

   % Índice de tablas
   % \listoftables

   \mainmatter
   \setlength{\parskip}{5pt}
       \chapter{Introducción}
    \section{Motivacion}
    La música siempre ha sido una de las aficiones más comunes entre los seres humanos.
    \section{Descripción del problema}
    
    \section{Objetivos}



       \chapter{Estado del Arte}
    \section{Software a desarrollar}
    \subsection{Aplicaciones de aprendizaje}
    Actualmente existen numerosas aplicaciones que ayudan a la adquisición de conocimientos de distintos temas, 
    como los idiomas, las matemáticas o la teoría para la conducción de un vehículo, entre otros. A continuación se
    detallarán las aplicaciones más destacadas.
   
    \subsubsection{Duolingo}
    \begin{wrapfigure}{r}{0.25\textwidth}
        \vspace*{-0.4cm}

        \centering
        \includegraphics[width=0.2\textwidth]{imagenes/c2/duolingo.png}
        \caption{Logo de Duolingo}
        \vspace*{-0.15cm}
    \end{wrapfigure}

    Duolingo es una de las aplicaciones más populares que existen para aprender idiomas. Esta aplicación se basa en el método
    de aprendizaje por inmersión, en el que el usuario estudia el idioma mediante una serie de actividades lúdicas.
    La aplicación se divide en lecciones o temas, las cuales contienen una serie de ejercicios que el usuario debe
    completar para ir avanzando por las distintas secciones. Estos ejercicios son muy variados y consisten en, por ejemplo,
    la traducción de palabras o frases con el teclado, la traducción de estas seleccionando bloques de palabras, la
    pronunciación de palabras o frases mediante el micrófono del dispositivo, la selección de imágenes, etc. Además,
    Duolingo ofrece en cada lección, antes de cada ejercicio, una breve explicación de la gramática o vocabulario con
    imágenes y explicaciones sencillas. Esta aplicación es gratuita (aunque tiene mejoras de pago dentro de esta) 
    y está disponible para dispositivos móviles (Android, iOS y Windows Phone).
 
    \subsubsection{FisicaMaster \& QuímicaMaster}

    \subsubsection{BMath}


    \subsubsection{Academons}

    \subsection{Aplicaciones de aprendizaje musical}
    Centrándonos ya en el tema que nos ocupa, el aprendizaje musical, existen numerosas aplicaciones que ayudan a la adquisición
    de conocimientos musicales o a la ayuda a aprender a tocar un instrumento musical. Algunas de estas aplicaciones son:

    \subsubsection{--}
    \subsubsection{--}
    \subsubsection{--}

    \subsubsection{Guitar Pro}

  



    \section{Desarrollo de Software}
\subsection{Flutter}
\subsection{Kotlin}
\subsection{React native}
\subsection{NodeJS}

\subsection{React}
\subsection{Angular}
\subsection{SQL}

   \chapter{Especificación de requisitos}


\section{Introducción}
A continuación, vamos a describir y detallar los requisitos del sistema 
que vamos a desarrollar. En esta sección se describirá qué vamos a construir y cómo lo vamos
a hacer, con sus restricciones específicas.

\subsection{Propósito}
En este capítulo se pretende describir de forma clara y precisa las funciones, carasterísticas y restricciones 
del sistema que se va a desarrollar. Estas definiciones servirán al equipo de desarrolo para 
conocer las necesidades del sistema y a los usuarios finales. 

Además, este documento servirá como base para el desarrollo de las funcionalidades descritas y como medio de comunicación 
entre las partes involucradas en el proyecto.
\subsection{Ámbito del Sistema}
El sistema que vamos a desarrollar es una aplicación multiplataforma que permitirá a los usuarios
aprender conceptos musicales de una forma amena y divertida.

\subsection{Definiciones, Acrónimos y Abreviaturas}

\begin{itemize}
    \item \textbf{Requisito:} Es una condición o característica que debe cumplir el sistema para satisfacer una necesidad o cumplir una función.
    \item \textbf{Funcionalidad:} Descripción de lo que debe hacer el producto software.
    \item \textbf{Restricción:} Condición que limita la funcionalidad del sistema.
    \item \textbf{Interfaces externas:} 
    \item \textbf{Rendimiento: }
    \item \textbf{Usuarios/Cliente: }
    \item \textbf{}
\end{itemize}

\subsection{Referencias}
\subsection{Visión General del Documento}


\section{Descripción General}
\subsection{Perspectiva del producto}
\subsection{Funciones del producto}
\subsection{Características de los usuarios}
\subsection{Restricciones}
\subsection{Suposiciones y dependencias}


\section{Requisitos Específicos}
\subsection{Requisitos funcionales}
\subsection{Requisitos no funcionales}
\subsection{Requisitos de información}
\subsection{Bocetos de interfaz de usuario}

   \chapter{Planificacion}
\section{Introducción}
En este capítulo vamos a realizar la planificación para el desarrollo del proyecto, la cual va a servir para poder
realizar un control de los avances del producto y asegurar que se cumplan los objetivos marcados en cada sprint (o iteración).
Para ello, se utilizarán algunos elementos de la metodología SCRUM, como son los sprints, las historias de usuario y el product backlog, etc. Es importante esto último
porque solo usaremos algunos principios y conceptos de metodologías ágiles pero no vamos a seguir al pie de la letra ninguna de ellas debido
a las condiciones en las que se va a desarrollar el proyecto (no hay un cliente real, no hay un equipo de desarrollo real, etc.).

\section{Velocidad}
A continuación vamos a estimar la velocidad inicial de trabajo que se tendrá en el desarrollo. Esta será una aproximación que nos ayudará a estimar la cantidad de trabajo que se debería realizar
en cada iteración. Sin embargo, esta podrá variar y ajustarse a lo largo del proyecto en función de la cantidad de trabajo que se logre completar en cada sprint.

Para calcular la velocidad, vamos a considerar las horas de trabajo que se pretenden dedicar al proyecto cada día y tratar de estimar cuanta cantidad de trabajo se podría realizar.

\begin{itemize}
\item Podríamos decir que partimos de un "equipo de desarrollo" de 1 miembro.
\item Se pretende dedicar 5 horas al día de trabajo aproximadamente.
\item Cada Sprint dura 2 semanas (14 días). Si cada día trabajamos 5 horas aproximadamente, obtenemos un total de 70 horas en cada Sprint, que equivalen a 3 días de trabajo reales.
\item En mi entorno, se estima que 1 PH es un día de trabajo ideal. Esto quiere decir que en cada jornada de trabajo real, se debería completar 1 PH. Sin embargo, en la realidad, esto no siempre es así ya que es una aproximación.
\item Si multiplicamos un programador por los 3 días de trabajo reales, obtenemos que deberíamos completar \textbf{3 PH} por sprint aproximadamente.
\end{itemize}

\textit{Nota: La duración de los Sprints puede variar en función de factores externos al proyecto. Pero se tratará de que duren 2 semanas para seguir un ritmo de trabajo constante.}

\textit{Nota 2: Puesto que estamos realizando estimaciones y suposiciones para la planificación, esta está sujeta a cambios a lo largo del proyecto.}
\section{Product Backlog}
A continuación, se listarán todas las historias de usuario ordenadas por prioridad de acuerdo a la importancia que tienen para el usuario final del producto.
\begin{itemize}
    \item \textbf{HU1 - } Como usuario quiero registrarme en la aplicación para poder comenzar a utilizar sus funcionalidades. (Prioridad: Alta | Puntos de Historia: 1)
    \item \textbf{HU2 - } Como usuario quiero iniciar sesión en la aplicación para poder acceder a sus funcionalidades. (Prioridad: Alta | Puntos de Historia: 1)
    \item \textbf{HU6 - } Como usuario quiero seleccionar una lección para leer el temario. (Prioridad: Alta | Puntos de Historia: 0.5)
    \item \textbf{HU7 - } Como usuario quiero empezar un test de una lección. (Prioridad: Alta | Puntos de Historia: 0.5)
    \item \textbf{HU26 - } Como usuario quiero responder una pregunta de un test del tipo escritura de texto. (Prioridad: Alta | Puntos de Historia: 0.5)
    \item \textbf{HU8 - } Como usuario quiero responder una pregunta de un test del tipo selección múltiple. (Prioridad: Alta | Puntos de Historia: 0.5)
    \item \textbf{HU9 - } Como usuario quiero responder una pregunta de un test del tipo selección única. (Prioridad: Alta | Puntos de Historia: 0.5)
    \item \textbf{HU10 - } Como usuario quiero responder una pregunta de un test del tipo respuesta por micrófono. (Prioridad: Alta | Puntos de Historia: 3)
    \item \textbf{HU11 - } Como usuario quiero ver el resultado de un test. (Prioridad: Media | Puntos de Historia: 0.5)
    \item \textbf{HU13 - } Como usuario quiero ver mi progreso en las lecciones. (Prioridad: Media | Puntos de Historia: 0.5)
    \item \textbf{HU28 - } Como profesor quiero ver una lista de todas las lecciones. (Prioridad: Media | Puntos de Historia: 0.5)
    \item \textbf{HU14 - } Como profesor quiero crear una lección. (Prioridad: Media | Puntos de Historia: 0.5)
    \item \textbf{HU15 - } Como profesor quiero modificar el texto de una lección. (Prioridad: Media | Puntos de Historia: 0.5)
    \item \textbf{HU16 - } Como profesor quiero añadir contenido multimedia a una lección. (Prioridad: Media | Puntos de Historia: 1)
    \item \textbf{HU17 - } Como profesor quiero eliminar contenido multimedia de una lección. (Prioridad: Media | Puntos de Historia: 0.5)
    \item \textbf{HU27 - } Como profesor quiero añadir preguntas a un test del tipo escritura de texto. (Prioridad: Media | Puntos de Historia: 0.5)
    \item \textbf{HU18 - } Como profesor quiero añadir preguntas a un test del tipo selección múltiple. (Prioridad: Media | Puntos de Historia: 0.5)
    \item \textbf{HU19 - } Como profesor quiero añadir preguntas a un test del tipo selección única. (Prioridad: Media | Puntos de Historia: 0.5)
    \item \textbf{HU20 - } Como profesor quiero añadir preguntas a un test del tipo respuesta por micrófono. (Prioridad: Media | Puntos de Historia: 1)
    \item \textbf{HU21 - } Como profesor quiero eliminar preguntas de un test. (Prioridad: Media | Puntos de Historia: 0.25)
    \item \textbf{HU22 - } Como profesor quiero modificar preguntas de un test. (Prioridad: Media | Puntos de Historia: 0.5)
    \item \textbf{HU24 - } Como administrador quiero modificar los datos de un usuario. (Prioridad: Media | Puntos de Historia: 1)
    \item \textbf{HU25 - } Como administrador quiero eliminar un usuario. (Prioridad: Media | Puntos de Historia: 0.5)
    \item \textbf{HU23 - } Como administrador quiero crear un usuario. (Prioridad: Baja | Puntos de Historia: 1)
    \item \textbf{HU12 - } Como usuario quiero ver el ranking de usuarios. (Prioridad: Baja | Puntos de Historia: 0.5)
    \item \textbf{HU5 - } Como usuario quiero ver el perfil de otros usuarios.  (Prioridad: Baja | Puntos de Historia: 0.5)
    \item \textbf{HU3 - } Como usuario quiero ver los datos personales y los logros de mi perfil. (Prioridad: Media | Puntos de Historia: 0.5)
    \item \textbf{HU4 - } Como usuario quiero editar los datos de mi perfil. (Prioridad: Media | Puntos de Historia: 1)
    \item \textbf{HU28 - } Como profesor quiero ver una lista de todas las lecciones. (Prioridad: Media | Puntos de Historia: 0.5)


\end{itemize}

\section{Sprints}
Nuestro proyecto se va a desarrollar en distintos Sprints, que son iteraciones de trabajo que se van a realizar en el proyecto.
Cada Sprint tendrá una duración de 2 semanas, y se pretenden realizar
\subsection{Sprint \#1 - Documentación inicial}
\textit{24/11/2022   -   08/12/2022}

En este Sprint se realizará:
\begin{itemize}

    \item Definición del proyecto y del alcance.
    \item Redactar Capítulo 1 - Introducción.
    \item Redactar Capítulo 2 - Estado del Arte.
\end{itemize}
\subsection{Sprint \#2 - Documentación de requisitos}
En este Sprint se planea realizar:
\textit{08/12/2022   -   12/01/2023}
\begin{itemize}
    \item Revisión de la documentación inicial y corrección de errores.
    \item Redactar Capítulo 3 - Especificación de Requisitos.
\end{itemize}

\subsection{Sprint \#3 - Documentación de HU}
\textit{12/01/2023   -   03/02/2023}
En este Sprint se realizará:
\begin{itemize}
    \item Revisión de la documentación de requisitos y corrección de errores.
    \item Redactar Capítulo 4 - Planificación.
    \item Redactar Capítulo 5 - Análisis del problema.
\end{itemize}
\subsection{Sprint \#4 - Diseño y comienzo del desarrollo}
\textit{03/02/2023   -   17/02/2023}
En este Sprint se terminará la documentación y comenzará el desarrollo con:
\begin{itemize}
    \item Redactar Capítulo 6 - Diseño.
    \item Preparación del backend para el desarrollo.
    \begin{itemize}
        \item Creación de la base de datos.
        \item Creación de los modelos, rutas y controladores del servidor con NodeJS.
    \end{itemize}
    \item Preparación del frontend para el desarrollo (creación de la aplicación de Flutter).
\end{itemize}
\subsection{Sprint \#5 - Registro y login de usuarios}
\textit{17/02/2023   -   03/03/2023}
\begin{itemize}
    \item Realizar la HU1 - Como alumno quiero registrarme en la aplicación. (Prioridad: Alta | Puntos de Historia: 1)
    \item Realizar la HU2 - Como alumno quiero iniciar sesión en la aplicación. (Prioridad: Alta | Puntos de Historia: 1)
    \item Realizar la HU6 - Como usuario quiero seleccionar una lección para leer el temario. (Prioridad: Alta | Puntos de Historia: 0.5)
    \item Realizar la HU7 - Como usuario quiero empezar un test de una lección. (Prioridad: Alta | Puntos de Historia: 0.5)
\end{itemize}


\subsection{Sprint \#6 - Tests y progreso}
\textit{03/03/2023   -   17/03/2023}
\begin{itemize}
    \item Realizar la HU26 - Como usuario quiero responder una pregunta de un test del tipo escritura de texto. (Prioridad: Alta | Puntos de Historia: 0.5)
    \item Realizar la HU8 - Como usuario quiero responder una pregunta de un test del tipo selección múltiple. (Prioridad: Alta | Puntos de Historia: 0.5)
    \item Realizar la HU9 - Como usuario quiero responder una pregunta de un test del tipo selección única. (Prioridad: Alta | Puntos de Historia: 0.5)
    \item Realizar la HU11 - Como usuario quiero ver el resultado de un test. (Prioridad: Media | Puntos de Historia: 0.5)
    \item Realizar la HU12 - Como usuario quiero ver mi progreso en las lecciones. (Prioridad: Media | Puntos de Historia: 0.5)

\end{itemize}

\subsection{Sprint \#7 - Entrada de micrófono y detección de tono}
\textit{17/03/2023   -   31/03/2023}
\begin{itemize}
    \item Realizar la HU10 - Como usuario quiero responder una pregunta de un test del tipo respuesta por micrófono. (Prioridad: Alta | Puntos de Historia: 3)
\end{itemize}

\subsection{Sprint \#8 - Funcionalidades del profesor con lecciones}
\textit{31/03/2023   -   14/04/2023}
\begin{itemize}

\item Realizar la HU13 - Como usuario quiero ver mi progreso en las lecciones. (Prioridad: Media | Puntos de Historia: 0.5)
\item Realizar la HU28 - Como profesor quiero ver una lista de todas las lecciones. (Prioridad: Media | Puntos de Historia: 0.5)
\item Realizar la HU14 - Como profesor quiero crear una lección. (Prioridad: Media | Puntos de Historia: 0.5)
\item Realizar la HU15 - Como profesor quiero modificar el texto de una lección. (Prioridad: Media | Puntos de Historia: 0.5)
\item Realizar la HU16 - Como profesor quiero añadir contenido multimedia a una lección. (Prioridad: Media | Puntos de Historia: 1)
\item Realizar la HU17 - Como profesor quiero eliminar contenido multimedia de una lección. (Prioridad: Media | Puntos de Historia: 0.5)
\end{itemize}

\subsection{Sprint \#9 - Funcionalidades del profesor con tests}
\textit{14/04/2023   -   28/04/2023}
\begin{itemize}
\item Realizar la HU27 - Como profesor quiero añadir preguntas a un test del tipo escritura de texto. (Prioridad: Media | Puntos de Historia: 0.5)
\item Realizar la HU18 - Como profesor quiero añadir preguntas a un test del tipo selección múltiple. (Prioridad: Media | Puntos de Historia: 0.5)
\item Realizar la HU19 - Como profesor quiero añadir preguntas a un test del tipo selección única. (Prioridad: Media | Puntos de Historia: 0.5)
\item Realizar la HU20 - Como profesor quiero añadir preguntas a un test del tipo respuesta por micrófono. (Prioridad: Media | Puntos de Historia: 1)
\item Realizar la HU21 - Como profesor quiero eliminar preguntas de un test. (Prioridad: Media | Puntos de Historia: 0.25)
\item Realizar la HU22 - Como profesor quiero modificar preguntas de un test. (Prioridad: Media | Puntos de Historia: 0.5)
\end{itemize}

\subsection{Sprint \#10 - Funcionalidades del administrador y ranking}
\textit{28/04/2023   -   12/05/2023}
\begin{itemize}
\item Realizar la HU24 - Como administrador quiero modificar los datos de un usuario. (Prioridad: Media | Puntos de Historia: 1)
\item Realizar la HU25 - Como administrador quiero eliminar un usuario. (Prioridad: Media | Puntos de Historia: 0.5)
\item Realizar la HU23 - Como administrador quiero crear un usuario. (Prioridad: Baja | Puntos de Historia: 1)
\item Realizar la HU12 - Como usuario quiero ver el ranking de usuarios. (Prioridad: Baja | Puntos de Historia: 0.5)
\end{itemize}

\subsection{Sprint \#11 - Pruebas y finalización del proyecto}
\textit{12/05/2023   -   26/05/2023}
\begin{itemize}
    \item Realizar la HU5 - Como usuario quiero ver el perfil de otros usuarios.  (Prioridad: Baja | Puntos de Historia: 0.5)
    \item Realizar la HU3 - Como usuario quiero ver los datos personales y los logros de mi perfil. (Prioridad: Media | Puntos de Historia: 0.5)
    \item Realizar la HU4 - Como usuario quiero editar los datos de mi perfil. (Prioridad: Media | Puntos de Historia: 1)
    \end{itemize}
    
    \subsection{Sprint \#12 - Pruebas, conclusión y finalización del proyecto}
\textit{26/05/2023   -   09/06/2023}
\begin{itemize}

    \item Redactar Capítulo 7 - Implementación
    \item Realizar mejoras de código.

\end{itemize}
\subsection{Sprint \#13 - Pruebas y mejoras}
\textit{09/06/2023   -   23/06/2023}
\begin{itemize}
    \item Realizar pruebas de integración.
    \item Realizar pruebas de unidad.
    \end{itemize}

\subsection{Sprint \#14 - Pruebas, conclusión y finalización del proyecto}
\textit{23/06/2023   -   15/07/2023}
\begin{itemize}

    \item Redactar Capítulo 8 - Pruebas.
    \item Redactar Capítulo 9 - Conclusiones.
    \item Revisión de todo el documento y corrección de errores.
\end{itemize}



\section{Diagrama de Gantt}


\begin{figure}[H]

    \centering
    \centerline{\includegraphics[width=1.25\textwidth]{imagenes/c4/gantt.png}}
    \caption{Diagrama de Gantt de la planificación del proyecto, divivido por Sprints.}
    \label{fig:artly}
    
    
\end{figure}

   \chapter{Análisis del problema}


\section{Historias de Usuario}

\begin{table}[h]
    \centering
    \resizebox{\textwidth}{!}{%
        \begin{tabular}{|
                >{\columncolor[HTML]{D8D8D8}}l |l|
                >{\columncolor[HTML]{D8D8D8}}l |l|l|l|}
            \hline
            \textbf{ID}                                                                  & HU1                                                                                                                                                                                                                                      & \textbf{Nombre}       & \multicolumn{3}{l|}{\begin{tabular}[c]{@{}l@{}}Como usuario quiero registrarme en la aplicación para poder comenzar \\ a utilizar sus funcionalidades.\end{tabular}}                                                                                                                                                                                                \\ \hline
            \textbf{PH}                                                                  & 1                                                                                                                                                                                                                                        & \textbf{Descripcion}  & \multicolumn{3}{l|}{\begin{tabular}[c]{@{}l@{}}El usuario podrá registrarse en el sistema rellenando un formulario con campos \\ como el nombre de usuario, el correo electrónico, la contraseña... \end{tabular}}                                                                                                                                                  \\ \hline
            \textbf{Prioridad}                                                           & 1                                                                                                                                                                                                                                        & \textbf{Dependencias} & \begin{tabular}[c]{@{}l@{}} --- \end{tabular}                                                                                                                                                                      & \cellcolor[HTML]{D8D8D8}\begin{tabular}[c]{@{}l@{}}\textbf{Requisitos Funcionales}\end{tabular} & \begin{tabular}[c]{@{}l@{}}RF 1\end{tabular} \\ \hline
            \multicolumn{3}{|l|}{\cellcolor[HTML]{D8D8D8}\textbf{Pruebas de aceptacion}} & \multicolumn{3}{l|}{\begin{tabular}[c]{@{}l@{}}1. Todos los campos rellenados por el usuario serán almacenados en la \\ base de datos \\ 2. La contraseña se almacenará cifrada para no comprometer la seguridad \\ del usuario \end{tabular}}                                                                                                                                                                                                                                                                                                                                                                                               \\ \hline
        \end{tabular}%
    }
\end{table}


\begin{table}[h]
    \centering
    \resizebox{\textwidth}{!}{%
        \begin{tabular}{|
                >{\columncolor[HTML]{D8D8D8}}l |l|
                >{\columncolor[HTML]{D8D8D8}}l |l|l|l|}
            \hline
            \textbf{ID}                                                                  & HU2                                                                                                                                                                                                                                      & \textbf{Nombre}       & \multicolumn{3}{l|}{\begin{tabular}[c]{@{}l@{}}Como usuario quiero iniciar sesión en la aplicación para poder \\ acceder a sus funcionalidades.\end{tabular}}                                                                                                                                                                                                \\ \hline
            \textbf{PH}                                                                  & 1                                                                                                                                                                                                                                        & \textbf{Descripcion}  & \multicolumn{3}{l|}{\begin{tabular}[c]{@{}l@{}}El usuario podrá registrarse en el sistema rellenando un formulario con campos \\ como el nombre de usuario, el correo electrónico, la contraseña... \end{tabular}}                                                                                                                                                  \\ \hline
            \textbf{Prioridad}                                                           & 1                                                                                                                                                                                                                                        & \textbf{Dependencias} & \begin{tabular}[c]{@{}l@{}} --- \end{tabular}                                                                                                                                                                      & \cellcolor[HTML]{D8D8D8}\begin{tabular}[c]{@{}l@{}}\textbf{Requisitos Funcionales}\end{tabular} & \begin{tabular}[c]{@{}l@{}}RF 1\end{tabular} \\ \hline
            \multicolumn{3}{|l|}{\cellcolor[HTML]{D8D8D8}\textbf{Pruebas de aceptacion}} & \multicolumn{3}{l|}{\begin{tabular}[c]{@{}l@{}}1. Todos los campos rellenados por el usuario serán almacenados en la base de datos \\ 2. La contraseña se almacenará cifrada para no comprometer la seguridad del usuario \end{tabular}}                                                                                                                                                                                                                                                                                                                                                                                               \\ \hline
        \end{tabular}%
    }
\end{table}


\begin{table}[h]
    \centering
    \resizebox{\textwidth}{!}{%
        \begin{tabular}{|
                >{\columncolor[HTML]{D8D8D8}}l |l|
                >{\columncolor[HTML]{D8D8D8}}l |l|l|l|}
            \hline
            \textbf{ID}                                                                  & HU1                                                                                                                                                                                                                                      & \textbf{Nombre}       & \multicolumn{3}{l|}{\begin{tabular}[c]{@{}l@{}}Como usuario quiero registrarme en la aplicación para poder comenzar a utilizar \\ sus funcionalidades.\end{tabular}}                                                                                                                                                                                                \\ \hline
            \textbf{PH}                                                                  & 1                                                                                                                                                                                                                                        & \textbf{Descripcion}  & \multicolumn{3}{l|}{\begin{tabular}[c]{@{}l@{}}El usuario podrá registrarse en el sistema rellenando un formulario con campos \\ como el nombre de usuario, el correo electrónico, la contraseña... \end{tabular}}                                                                                                                                                  \\ \hline
            \textbf{Prioridad}                                                           & 1                                                                                                                                                                                                                                        & \textbf{Dependencias} & \begin{tabular}[c]{@{}l@{}} --- \end{tabular}                                                                                                                                                                      & \cellcolor[HTML]{D8D8D8}\begin{tabular}[c]{@{}l@{}}\textbf{Requisitos Funcionales}\end{tabular} & \begin{tabular}[c]{@{}l@{}}RF 1\end{tabular} \\ \hline
            \multicolumn{3}{|l|}{\cellcolor[HTML]{D8D8D8}\textbf{Pruebas de aceptacion}} & \multicolumn{3}{l|}{\begin{tabular}[c]{@{}l@{}}1. Todos los campos rellenados por el usuario serán almacenados en la base de datos \\ 2. La contraseña se almacenará cifrada para no comprometer la seguridad del usuario \end{tabular}}                                                                                                                                                                                                                                                                                                                                                                                               \\ \hline
        \end{tabular}%
    }
\end{table}


\begin{table}[h]
    \centering
    \resizebox{\textwidth}{!}{%
        \begin{tabular}{|
                >{\columncolor[HTML]{D8D8D8}}l |l|
                >{\columncolor[HTML]{D8D8D8}}l |l|l|l|}
            \hline
            \textbf{ID}                                                                  & HU1                                                                                                                                                                                                                                      & \textbf{Nombre}       & \multicolumn{3}{l|}{\begin{tabular}[c]{@{}l@{}}Como usuario quiero registrarme en la aplicación para poder comenzar a utilizar \\ sus funcionalidades.\end{tabular}}                                                                                                                                                                                                \\ \hline
            \textbf{PH}                                                                  & 1                                                                                                                                                                                                                                        & \textbf{Descripcion}  & \multicolumn{3}{l|}{\begin{tabular}[c]{@{}l@{}}El usuario podrá registrarse en el sistema rellenando un formulario con campos \\ como el nombre de usuario, el correo electrónico, la contraseña... \end{tabular}}                                                                                                                                                  \\ \hline
            \textbf{Prioridad}                                                           & 1                                                                                                                                                                                                                                        & \textbf{Dependencias} & \begin{tabular}[c]{@{}l@{}} --- \end{tabular}                                                                                                                                                                      & \cellcolor[HTML]{D8D8D8}\begin{tabular}[c]{@{}l@{}}\textbf{Requisitos Funcionales}\end{tabular} & \begin{tabular}[c]{@{}l@{}}RF 1\end{tabular} \\ \hline
            \multicolumn{3}{|l|}{\cellcolor[HTML]{D8D8D8}\textbf{Pruebas de aceptacion}} & \multicolumn{3}{l|}{\begin{tabular}[c]{@{}l@{}}1. Todos los campos rellenados por el usuario serán almacenados en la base de datos \\ 2. La contraseña se almacenará cifrada para no comprometer la seguridad del usuario \end{tabular}}                                                                                                                                                                                                                                                                                                                                                                                               \\ \hline
        \end{tabular}%
    }
\end{table}


\begin{table}[h]
    \centering
    \resizebox{\textwidth}{!}{%
        \begin{tabular}{|
                >{\columncolor[HTML]{D8D8D8}}l |l|
                >{\columncolor[HTML]{D8D8D8}}l |l|l|l|}
            \hline
            \textbf{ID}                                                                  & HU1                                                                                                                                                                                                                                      & \textbf{Nombre}       & \multicolumn{3}{l|}{\begin{tabular}[c]{@{}l@{}}Como usuario quiero registrarme en la aplicación para poder comenzar a utilizar \\ sus funcionalidades.\end{tabular}}                                                                                                                                                                                                \\ \hline
            \textbf{PH}                                                                  & 1                                                                                                                                                                                                                                        & \textbf{Descripcion}  & \multicolumn{3}{l|}{\begin{tabular}[c]{@{}l@{}}El usuario podrá registrarse en el sistema rellenando un formulario con campos \\ como el nombre de usuario, el correo electrónico, la contraseña... \end{tabular}}                                                                                                                                                  \\ \hline
            \textbf{Prioridad}                                                           & 1                                                                                                                                                                                                                                        & \textbf{Dependencias} & \begin{tabular}[c]{@{}l@{}} --- \end{tabular}                                                                                                                                                                      & \cellcolor[HTML]{D8D8D8}\begin{tabular}[c]{@{}l@{}}\textbf{Requisitos Funcionales}\end{tabular} & \begin{tabular}[c]{@{}l@{}}RF 1\end{tabular} \\ \hline
            \multicolumn{3}{|l|}{\cellcolor[HTML]{D8D8D8}\textbf{Pruebas de aceptacion}} & \multicolumn{3}{l|}{\begin{tabular}[c]{@{}l@{}}1. Todos los campos rellenados por el usuario serán almacenados en la base de datos \\ 2. La contraseña se almacenará cifrada para no comprometer la seguridad del usuario \end{tabular}}                                                                                                                                                                                                                                                                                                                                                                                               \\ \hline
        \end{tabular}%
    }
\end{table}


\begin{table}[h]
    \centering
    \resizebox{\textwidth}{!}{%
        \begin{tabular}{|
                >{\columncolor[HTML]{D8D8D8}}l |l|
                >{\columncolor[HTML]{D8D8D8}}l |l|l|l|}
            \hline
            \textbf{ID}                                                                  & HU1                                                                                                                                                                                                                                      & \textbf{Nombre}       & \multicolumn{3}{l|}{\begin{tabular}[c]{@{}l@{}}Como usuario quiero registrarme en la aplicación para poder comenzar a utilizar \\ sus funcionalidades.\end{tabular}}                                                                                                                                                                                                \\ \hline
            \textbf{PH}                                                                  & 1                                                                                                                                                                                                                                        & \textbf{Descripcion}  & \multicolumn{3}{l|}{\begin{tabular}[c]{@{}l@{}}El usuario podrá registrarse en el sistema rellenando un formulario con campos \\ como el nombre de usuario, el correo electrónico, la contraseña... \end{tabular}}                                                                                                                                                  \\ \hline
            \textbf{Prioridad}                                                           & 1                                                                                                                                                                                                                                        & \textbf{Dependencias} & \begin{tabular}[c]{@{}l@{}} --- \end{tabular}                                                                                                                                                                      & \cellcolor[HTML]{D8D8D8}\begin{tabular}[c]{@{}l@{}}\textbf{Requisitos Funcionales}\end{tabular} & \begin{tabular}[c]{@{}l@{}}RF 1\end{tabular} \\ \hline
            \multicolumn{3}{|l|}{\cellcolor[HTML]{D8D8D8}\textbf{Pruebas de aceptacion}} & \multicolumn{3}{l|}{\begin{tabular}[c]{@{}l@{}}1. Todos los campos rellenados por el usuario serán almacenados en la base de datos \\ 2. La contraseña se almacenará cifrada para no comprometer la seguridad del usuario \end{tabular}}                                                                                                                                                                                                                                                                                                                                                                                               \\ \hline
        \end{tabular}%
    }
\end{table}
\section{Diagrama de Clases}

   
   \chapter{Diseño}

\section{Introducción}
En este capítulo se realizará el diseño del software a desarrollar. Para ello, se detallará cómo se va a desarrollar la aplicación, qué arquitectura y tecnologías se van a utilizar y cómo se va a realizar el diseño de la interfaz de usuario. 
Esta fase tiene como objetivo definir la estructura del sistema y la función de cada una de sus partes, lo cual permitirá que el desarrollo sea más eficiente y que el resultado final tenga mayor calidad.

\subsection{Arquitectura del sistema}
Para el desarrollo de la aplicación se ha decidido utilizar una arquitectura \textbf{cliente-servidor}, donde el servidor se encargará de gestionar la base de datos y de realizar las operaciones necesarias para extraer la información de dicho modelo, mientras que el cliente será 
la aplicación móvil que se encargará de mostrar la información al usuario y de realizar las operaciones que el usuario solicite. Además, también será una arquitectura basada en el patrón de diseño \textbf{Modelo / Vista / Controlador (MVC)}, donde el modelo gestionará la información
de la base de datos, la vista se hará cargo de mostrar la información al usuario y facilitar la interacción de este con el sistema y el controlador se encargará de gestionar las operaciones que el usuario solicite. Todo esto mediante las tecnologías \textbf{Node.js} para el controlador,  \textbf{MongoDB} (y mongoose) para
el modelo y Flutter para las distintas vistas que los usuarios tendrán. 

\begin{figure}[H]
    \centering
    \centerline{\includegraphics[width=\textwidth]{imagenes/c6/arch.png}}
    \caption{Diagrama de la arquitectura del sistema, donde se muestra la comunicación entre el servidor y la aplicación móvil.}
    \label{fig:diagramadearquitectura}    
\end{figure}

\subsection{Diagrama de base de datos}

\begin{figure}[H]
    \centering
    \centerline{\includegraphics[width=\textwidth]{imagenes/c6/bbdd.png}}
    \caption{Diagrama de base de datos del sistema, donde se muestran los documentos que se van a almacenar en nuestra base de datos de MongoDB.}
    \label{fig:diagramadearquitectura}    
\end{figure}


\subsection{Diagrama de clases}
A continuación se muestra el diagrama de clases de diseño, el cual se ha realizado a partir del diagrama de clases de análisis del capítulo anterior En esta ocasión se han detallado más
los atributos y los métodos de cada clase, facilitando así la comprensión de las relaciones entre las clases y de la estructura del sistema.
\begin{figure}[H]
    \centering
    \centerline{\includegraphics[width=\textwidth]{imagenes/c6/diagramadeclases.png}}
    \caption{Diagrama de clases del sistema donde se detallan las propiedades y las relaciones de las distintas clases o entidades que tendrá el software.}
    \label{fig:diagramadearquitectura}    
\end{figure}

\subsection{Diagramas de secuencia}
En lo que respecta a los diagramas de secuencia, se han realizado algunos ejemplos de cómo sería el flujo de una operación en el sistema. Como hemos visto anteriormente,
se seguirá una arquitectura basada en el patrón de diseño MVC, por lo que las peticiones pasarán por la vista, luego por el controlador y finalmente por el modelo.
En este caso, se ha realizado un diagrama de secuencia de cómo sería el flujo de una operación en la que un usuario se registra en la aplicación, otro para un usuario que contesta un test y otro
para un profesor que modifica una lección.

\begin{figure}[H]
    \centering
    \centerline{\includegraphics[width=\textwidth]{imagenes/c6/diagramadesecuencia.png}}
    \caption{Diagrama de secuencia de un usuario registrandose, el cual interactuará con la vista para poder hacer la petición al controlador y guardar su usuario en el modelo.}
    \label{fig:diagramadesecuencia}    
\end{figure}


\subsection{Diseño de interfaces de usuario}
Por último, en este apartado se presentan los bocetos de las interfaces de usuario que se han diseñado para la aplicación. Estos bocetos se han realizado con la herramienta Canva y serán de ayuda para la realización del diseño final de las interfaces de usuario,
que, pese a que seguirán una estructura similar a la de los bocetos, podrán variar en algunos detalles y aspectos de diseño.

\begin{figure}[H]
    \centering
    \centerline{\includegraphics[width=0.55\textwidth, frame]{imagenes/c6/1.png}}
    \caption{Boceto de la pantalla principal de la aplicación donde se muestran las lecciones disponibles para el usuario.}
    \label{fig:pantallaprincipal}
    
    
\end{figure}

\begin{figure}[H]
    \centering
    \centerline{\includegraphics[width=0.55\textwidth, frame]{imagenes/c6/2.png}}
    \caption{Boceto de la pantalla del perfil de usuario con la sesión iniciada donde se muestran sus datos.}
    \label{fig:perfil}
    
    
\end{figure}


\begin{figure}[H]
    \centering
    \centerline{\includegraphics[width=0.55\textwidth, frame]{imagenes/c6/3.png}}
    \caption{Boceto de la pantalla de logros conseguidos por el usuario.}
    \label{fig:logros}
    
    
\end{figure}


\begin{figure}[H]
    \centering
    \centerline{\includegraphics[width=0.55\textwidth, frame]{imagenes/c6/4.png}}
    \caption{Boceto de la pantalla de una lección, con el texto, el contenido multimedia y el botón para comenzar el test.}
    \label{fig:leccion}
\end{figure}

\begin{figure}[H]
    \centering
    \centerline{\includegraphics[width=0.55\textwidth, frame]{imagenes/c6/5.png}}
    \caption{Boceto de la pantalla de una pregunta de test de tipo selección única.}
    \label{fig:seleccionunica}
\end{figure}

\begin{figure}[H]
    \centering
    \centerline{\includegraphics[width=0.55\textwidth, frame]{imagenes/c6/6.png}}
    \caption{Boceto de la pantalla de una pregunta de test de tipo selección multiple.}
    \label{fig:seleccionmultiple}
\end{figure}

\begin{figure}[H]
    \centering
    \centerline{\includegraphics[width=0.55\textwidth, frame]{imagenes/c6/7.png}}
    \caption{Boceto de la pantalla de una pregunta de test de tipo escritura de texto.}
    \label{fig:escrituratexto}
\end{figure}

\begin{figure}[H]
    \centering
    \centerline{\includegraphics[width=0.55\textwidth, frame]{imagenes/c6/8.png}}
    \caption{Boceto de la pantalla de una pregunta de test de tipo entrada por micrófono.}
    \label{fig:microfono}
\end{figure}

\begin{figure}[H]
    \centering
    \centerline{\includegraphics[width=0.55\textwidth, frame]{imagenes/c6/9.png}}
    \caption{Boceto de la pantalla de inicio de sesión, donde se pedirá al usuario el correo electrónico y la contraseña.}
    \label{fig:login}
\end{figure}

\begin{figure}[H]
    \centering
    \centerline{\includegraphics[width=0.55\textwidth, frame]{imagenes/c6/10.png}}
    \caption{Boceto de la pantalla de registro, donde se pedirá al usuario sus datos personales necesarios para crear la cuenta como el nombre, los apellidos, el correo y la contraseña.}
    \label{fig:registro}
\end{figure}
   
   \chapter{Implementación}
En este séptimo capítulo se presentará la implementación del sistema. En primer lugar, se presentará la estructura del proyecto que se ha realizado en la preparación del frontend y del backend y a continuación
se describirá la implementación de cada una de las funcionalidades del sistema, presentando la documentación de cada ruta de la API y describiendo el código escrito durante el desarrollo.

   
   \chapter{Pruebas}
\label{cap:pruebas}
En este capítulo se describen las pruebas realizadas al sistema, tanto de forma individual como de forma integrada. Se describen los casos de prueba, los resultados obtenidos y las conclusiones extraídas de los mismos.
El objetivo de las pruebas es comprobar que el sistema cumple con los requisitos establecidos en el capítulo (Figura \ref{cap:especificacion-requisitos}) y que funcione correctamente.

\section{Paquetes utilizados}
\label{sec:paquetes-utilizados}
Para realizar las pruebas se han utilizado los siguientes paquetes:

\begin{itemize}
    \item \textbf{flutter\_test}: paquete que contiene las clases necesarias para realizar las pruebas de unidad.
    \item \textbf{nock}: paquete que permite simular las peticiones HTTP realizadas por la aplicación y devolver una respuesta simulada ~\cite{nockpackage}. 
    \item \textbf{flutter\_driver}: paquete que contiene las clases necesarias para realizar las pruebas de integración. 
\end{itemize}


\section{Pruebas de unidad}
\label{sec:pruebas-unidad}
Las pruebas de unidad son aquellas que se realizan sobre los componentes más pequeños del sistema, como pueden ser las funciones o los métodos. Estas pruebas se realizan de forma individual, aislando el componente a probar de los demás componentes del sistema. 
De esta forma nos aseguramos de que el componente funciona correctamente y que no depende de otros componentes para su correcto funcionamiento.
\newpage



\subsection{Lecciones}
\label{subsec:pruebas-unidad-lecciones}
A continuación se describen las pruebas realizadas sobre las lecciones:
\begin{itemize}
    \item \textbf{Obtención y guardado}: Las lecciones deben obtenerse y guardarse en el vector del provider (método fetchAndSetLecciones). \textit{Tras la ejecución del método el provider debe contener una lista no vacía de lecciones.} 
    \item \textbf{Obtención del índice mediante el id}: Dado un id de una lección, se debe obtener el índice de la lección en el vector del provider (método getIndice). \textit{El método no debe devolver el valor '-1'.}
    \item \textbf{Creación de una lección}: Se debe crear una lección con los datos correctos (método crearLeccion). \textit{Tras ejecutar el método, el vector de lecciones del provider debe tener una longitud mayor que antes de ejecutarlo.}
    \item \textbf{Modificación de una lección}: Se debe modificar una lección con los datos correctos (método modificarLeccion). \textit{Tras ejecutar el método, los datos antiguos y nuevos de la lección deben ser distintos.}
    \item \textbf{Eliminación de una lección}: Se debe eliminar una lección coorectamente (método eliminarLeccion). \textit{Tras ejecutar el método, el vector de lecciones del provider debe tener una longitud menor que antes de ejecutarlo.}
\end{itemize}

\begin{figure}[H]
    \centering
    \includegraphics[width=\textwidth]{imagenes/c8/pruebalecciones.png}
    \caption{Resulados de las pruebas de unidad de las lecciones}
    \label{fig:pruebas_unidad_lecciones}
\end{figure}


\subsection{Preguntas}
\label{subsec:pruebas-unidad-preguntas}
Las pruebas realizadas sobre las preguntas son las siguientes:
\begin{itemize}
    \item \textbf{Obtención y guardado de preguntas del profesor}: Las preguntas de la vista del profesor deben obtenerse y guardarse en el vector del provider PreguntasProfesor (método fetchAndSetPreguntas). \textit{Después de ejecutar el método, el provider debe contener una lista no vacía de preguntas.}
    \item \textbf{Obtención del índice mediante el id}: Dado un id de una pregunta, se debe obtener el índice de la pregunta en el vector del provider (método getIndice). \textit{El método no debe devolver el valor '-1'.}
    \item \textbf{Creación de una pregunta}: Se debe crear una pregunta con los datos correctos (método crearPregunta). \textit{Tras ejecutar el método, el vector de preguntas del provider debe tener una longitud mayor que antes de ejecutarlo }
    \item \textbf{Modificación de una pregunta}: Se debe modificar una pregunta con los datos correctos (método modificarPregunta). \textit{Tras la ejución del método, los datos antiguos y nuevos de la pregunta deben ser distintos.}
    \item \textbf{Eliminación de una pregunta}: Se debe eliminar una pregunta coorectamente (método eliminarPregunta). \textit{Tras ejecutar el método, el provider no debe contener la pregunta.}
    \item \textbf{Obtención y guardado de preguntas}: Las preguntas deben obtenerse y guardarse en el vector del provider Preguntas (método fetchAndSetPreguntas). \textit{Después de ejecutar el método, el provider debe contener una lista no vacía de preguntas.}
    \item \textbf{Asignar opción}: Se debe asignar una opción a una pregunta correctamente cuando se crea o modifica una pregunta (método setValor). \textit{Tras ejecutar el método, las opciones de la pregunta deben coincidir con las opciones asignadas.}
    \item \textbf{Asignar respuesta}: Se debe asignar una respuesta a una pregunta correctamente cuando el usuario responde (método setRespuestas). \textit{Después de la ejecución, la respuesta de la pregunta debe coincidir con la respuesta asignada.}
    \item \textbf{Asignar pulsado}: Se debe asignar el valor de la variable "pulsado" correctamente cuando el usuario pulsa una opción (método setPulsado). \textit{Después de la ejecución, los valores del array "pulsado" deben coincidir con los asignados.}
\end{itemize}

\begin{figure}[H]
    \centering
    \includegraphics[width=\textwidth]{imagenes/c8/pruebapreguntas.png}
    \caption{Resulados de las pruebas de unidad de las preguntas}
    \label{fig:pruebas_unidad_preguntas}
\end{figure}


\subsection{Usuarios}
\label{subsec:pruebas-unidad-usuarios}
Las pruebas realizadas sobre los usuarios son las siguientes:
\begin{itemize}
    \item \textbf{Obtención y guardado}: Los usuarios deben obtenerse y guardarse en el vector del provider (método fetchAndSetUsuarios). \textit{Tras la ejecución del método el provider debe contener una lista no vacía de usuarios.}
    \item \textbf{Obtención del índice mediante el id}: Dado un id de un usuario, se debe obtener el índice del usuario en el vector del provider (método getIndice). \textit{El método no debe devolver el valor '-1'.}
    \item \textbf{Creación de un usuario}: Se debe crear un usuario con los datos correctos (método crearUsuario). \textit{Tras ejecutar el método, el vector de usuarios del provider debe tener una longitud mayor que antes de ejecutarlo.}
    \item \textbf{Modificación de un usuario}: Se debe modificar un usuario con los datos correctos (método modificarUsuario). \textit{Tras ejecutar el método, los datos antiguos y nuevos del usuario deben ser distintos.}
    \item \textbf{Eliminación de un usuario}: Se debe eliminar un usuario coorectamente (método eliminarUsuario). \textit{Tras ejecutar el método, el vector de usuarios del provider debe tener una longitud menor que antes de ejecutarlo.}
\end{itemize}

\begin{figure}[H]
    \centering
    \includegraphics[width=\textwidth]{imagenes/c8/pruebausuarios.png}
    \caption{Resulados de las pruebas de unidad de los usuarios}
    \label{fig:pruebas_unidad_usuarios}
\end{figure}

\subsection{Logros}
\label{subsec:pruebas-unidad-logros}
Las pruebas realizadas sobre los logros son las siguientes:
\begin{itemize}
    \item \textbf{Obtención y guardado}: Los logros deben obtenerse y guardarse en el vector del provider (método fetchAndSetLogros). \textit{Tras la ejecución del método el provider debe contener una lista no vacía de logros.}
    \item \textbf{Obtención del índice mediante el id}: Dado un id de un logro, se debe obtener el índice del logro en el vector del provider (método getIndice). \textit{El método no debe devolver el valor '-1'.}
    \item \textbf{Creación de un logro}: Se debe crear un logro con los datos correctos (método crearLogro). \textit{Tras ejecutar el método, el vector de logros del provider debe tener una longitud mayor que antes de ejecutarlo.}
    \item \textbf{Modificación de un logro}: Se debe modificar un logro con los datos correctos (método modificarLogro). \textit{Tras ejecutar el método, los datos antiguos y nuevos del logro deben ser distintos.}
    \item \textbf{Eliminación de un logro}: Se debe eliminar un logro coorectamente (método borrarLogro). \textit{Tras ejecutar el método, el vector de logros del provider debe tener una longitud menor que antes de ejecutarlo.}
\end{itemize}

\begin{figure}[H]
    \centering
    \includegraphics[width=\textwidth]{imagenes/c8/pruebalogros.png}
    \caption{Resulados de las pruebas de unidad de los logros}
    \label{fig:pruebas_unidad_logros}
\end{figure}


\section{Pruebas de controlador o widget}
\label{sec:pruebas-controlador}
Las pruebas de controlador o widget son aquellas que se realizan sobre los controladores o widgets de la aplicación (Figura \ref{fig:prueba_widgets}). Estas pruebas se realizan de forma individual, aislando el controlador o widget a probar de los demás componentes del sistema.
A continuación se muestran las pruebas realizadas sobre algunos widgets de la aplicación.

\subsection{Botón de inicio de sesión}
\label{subsec:pruebas-controlador-boton-inicio-sesión}
El botón de inicio de sesión se encuentra en la pantalla de inicio de la aplicación. Se ha realizado una prueba sobre el botón, comprobando que al pulsar el botón se accede a la pantalla de lecciones (pantalla principal de la aplicación).
El test verifica que al pulsar dicho botón aparece un texto perteneciente a la pantalla de lecciones, en este caso ``Los instrumentos''.

\subsection{Botón de acceso a lista de lecciones del profesor}
\label{subsec:pruebas-controlador-boton-lecciones}
El dashboard del usuario contiene un botón para acceder a las lecciones. Se ha realizado una prueba sobre el botón, comprobando que al pulsar el botón se accede a la pantalla de lista de lecciones. 
El test verifica que al pulsar dicho botón se muestra un texto perteneciente a la pantalla de lecciones, en este caso ``Crear Lección'' y ``Borrar''.

\subsection{Botón de acceso a lista de logros del profesor}
\label{subsec:pruebas-controlador-boton-logros}
El dashboard del usuario contiene un botón para acceder a los logros. Se ha realizado una prueba sobre el botón, comprobando que al pulsar el botón se accede a la pantalla de lista de logros. 
El test verifica que al pulsar dicho botón se muestra un texto perteneciente a la pantalla de logros, en este caso ``Crear Logro'' y ``Borrar'' (Figura \ref{fig:prueba_lista_logros}).
\begin{figure}[H]
    \centering
    \includegraphics[width=\textwidth]{imagenes/c8/pruebawidget1.png}
    \caption{Código de la prueba de controlador o widget de acceso a lista de logros}
    \label{fig:prueba_lista_logros}
\end{figure}


\subsection{Botón de acceso a lista de preguntas del profesor}
\label{subsec:pruebas-controlador-boton-preguntas}
El dashboard del usuario contiene un botón para acceder a las preguntas. Se ha realizado una prueba sobre el botón, comprobando que al pulsar el botón se accede a la pantalla de lista de preguntas. 
El test verifica que al pulsar dicho botón aparece un texto perteneciente a la pantalla de preguntas, en este caso ``Crear Pregunta'' y ``Borrar''.

\subsection{Botón de acceso a lista de usuarios }
\label{subsec:pruebas-controlador-boton-usuarios}
El dashboard del usuario contiene un botón para acceder a los usuarios. Se ha realizado una prueba sobre el botón, comprobando que al pulsar el botón se accede a la pantalla de lista de usuarios. 
El test verifica que al pulsar dicho botón se muestra un texto perteneciente a la pantalla de usuarios, en este caso ``Crear Usuario'' y ``Borrar''.

\subsection{Botón de acceso al perfil del usuario}
\label{subsec:pruebas-controlador-boton-perfil-usuario}
El drawer contiene un botón para acceder al perfil del usuario. Se ha realizado una prueba sobre el botón, comprobando que al pulsar el botón se accede a la pantalla de perfil del usuario.
El test verifica que al pulsar dicho botón aparece un texto perteneciente a la pantalla de perfil del usuario, en este caso ``Mis logros''.

\subsection{Botón de acceso a una lección}
\label{subsec:pruebas-controlador-boton-leccion}
Al pulsar sobre una lección de la lista de lecciones, se accede a la pantalla de lección. Se ha realizado una prueba sobre el botón, comprobando que dicho acceso se realiza correctamente.
El test verifica que al pulsar dicho botón aparece un texto perteneciente a la pantalla de lección, en este caso ``Historial''.

\subsection{Botón de acceso al inicio}
\label{subsec:pruebas-controlador-boton-leccion}
El drawer contiene un botón para acceder a la pantalla de lecciones. Se ha realizado una prueba sobre el botón, comprobando que dicho acceso se realiza correctamente.
El test verifica que al pulsar dicho botón se muestra un texto perteneciente a la pantalla de lecciones, en este caso ``Los instrumentos'' (Figura \ref{fig:prueba_lecciones}).

\begin{figure}[H]
    \centering
    \includegraphics[width=\textwidth]{imagenes/c8/pruebawidget2.png}
    \caption{Código de la prueba de controlador o widget de acceso a la pantalla de lecciones}
    \label{fig:prueba_lecciones}
\end{figure}

\subsection{Botón de acceso al dashboard}
\label{subsec:pruebas-controlador-boton-leccion}
El drawer contiene un botón para acceder al dashboard. Se ha realizado una prueba sobre el botón, comprobando que dicho acceso se realiza correctamente.
El test verifica si al pulsar dicho botón encuentra un texto perteneciente a la pantalla de dashboard, en este caso ``Lecciones". 


\begin{figure}[H]
    \centering
    \includegraphics[width=\textwidth]{imagenes/c8/pruebawidgets.png}
    \caption{Resultados de las pruebas de controlador o widget}
    \label{fig:prueba_widgets}
\end{figure}


\section{Pruebas de sistema o de integración}
\label{sec:pruebas-sistema}
Las pruebas de sistema o de integración son aquellas que se realizan sobre el sistema completo. Estas pruebas tratan de verificar el flujo de uso de una determinada funcionalidad de la aplicación.
De esta forma nos aseguramos de que el sistema funciona de forma adecuada simulando varias acciones que podría realizar un usuario.

\subsection{Creación de usuario}
En este test se comprueba que se crea un usuario correctamente. Para ello se siguen los siguientes pasos:
\begin{enumerate}
    \item Se comienza en la pantalla de inicio de sesión.
    \item Se rellenan los campos de inicio de sesión con los datos del profesor.
    \item Se pulsa el botón de inicio de sesión.
    \item Se abre el drawer y se pulsa el botón de acceso a la dashboard.
    \item Se pulsa el botón de acceso a la gestión de Usuarios.
    \item Se pulsa el botón de crear usuario.
    \item Se rellenan los campos de creación de usuario con los datos del alumno.
    \item Se pulsa el botón de crear usuario.
    \item Se comprueba que se ha creado el usuario.
\end{enumerate}

\begin{figure}[H]
    \centering
    \includegraphics[width=\textwidth]{imagenes/c8/testint1.png}
    \caption{Pantallas por las que pasa la prueba de creación de usuario}
    \label{fig:prueba_creacion_usuario}
\end{figure}

\newpage


\subsection{Edición de datos del perfil}

\begin{figure}[H]
    \centering
    \includegraphics[width=\textwidth]{imagenes/c8/testint2.png}
    \caption{Pantallas por las que pasa la prueba de modificación de datos del perfil}
    \label{fig:prueba_edicion_perfil}
\end{figure}

Esta prueba verifica que se editen los datos del perfil de un usuario (Figura \ref{fig:prueba_edicion_perfil}). Para ello se siguen los siguientes pasos:

\begin{enumerate}
    \item Se comienza en la pantalla de lista de lecciones. (la sesión ya está iniciada del test anterior)
    \item Se abre el drawer y se pulsa el botón de acceso al perfil del usuario.
    \item Se pulsa el botón de editar perfil.
    \item Se rellenan los campos de edición de perfil con los nuevos datos del profesor.
    \item Se pulsa el botón de editar perfil.
    \item Se comprueba que se han editado los datos del perfil.
\end{enumerate}



\subsection{Realización de un test}
Esta prueba verifica que se puede realizar un test (Figura \ref{fig:prueba_realizacion_test}) y contestar a las preguntas de este. Para ello se siguen los siguientes pasos:
\begin{enumerate}
    \item Se comienza en la pantalla de lista de lecciones.
    \item Se pulsa sobre la lección "Introducción" (por ejemplo).
    \item Se pulsa el botón de "Hacer Test".
    \item Se responde a las preguntas en función del tipo:
    \begin{enumerate}
        \item Si es de tipo "unica", se pulsa sobre la segunda opción.
        \item Si es de tipo "multiple", se pulsa sobre la segunda y la tercera opción.
        \item Si es de tipo "texto", se escribe una respuesta cualquiera. En este caso se ha escrito "melodía".
    \end{enumerate}
    \item Se pulsa el botón de "Enviar".
    \item Se comprueba que se ha realizado el test.
\end{enumerate}

\begin{figure}[H]
    \centering
    \includegraphics[width=0.85\textwidth]{imagenes/c8/testint3.png}
    \caption{Pantallas por las que pasa la prueba de realización de un test}
    \label{fig:prueba_realizacion_test}
\end{figure}

   
   \chapter{Conclusiones}
\label{cap:conclusiones}
A modo de cierre puedo decir que, tras meses de trabajo, se ha conseguido realizar la aplicación multiplataforma que se pretendía al comienzo del proyecto. 
Se ha logrado desarrollar un sistema para el aprendizaje musical de forma interactiva y basado en métodos de gamificación. En concreto, se han conseguido cumplir con éxito
los siguientes objetivos:
\begin{itemize}
    \item Se ha construido un sistema seguro y robusto, que permite a los usuarios registrarse y acceder a la aplicación con seguridad.
    \item Se ha desarrollado un sistema de gestión de usuarios, que permite a los usuarios guardar su progreso y acceder a él desde cualquier dispositivo.
    \item Se ha implementado un sistema de administración para el profesor y el administrador, que permite gestionar los usuarios, las lecciones, las preguntas y los logros de forma fácil y cómoda.
    \item Se ha desarrollado un sistema de lecciones, que permite a los usuarios aprender los conceptos básicos de la música de forma interactiva.
    \item Se ha implementado un sistema de preguntas, que permite a los usuarios poner a prueba sus conocimientos de forma interactiva. En concreto, uno de los mayores retos del proyecto ha sido la implementación de la funcionalidad de preguntas de tipo \textit{micrófono} y tras mucho
    esfuerzo se ha logrado desarrollar con un resultado satisfactorio.
    \item Se ha desarrollado un sistema de logros, que permite a los usuarios obtener recompensas por su progreso y sus logros.
\end{itemize}

El desarrollo de este proyecto ha sido una experiencia muy enriquecedora, ya que ha permitido aprender y poner en práctica muchos de los conocimientos adquiridos durante los cuatro años de grado. Además, me ha permitido afianzar mis habilidades de programación y aprender nuevas tecnologías y herramientas que no había utilizado anteriormente o de forma muy superficial.

\section{Valoración personal}
\label{sec:valoracion_personal}
Mi experiencia con este proyecto ha sido muy positiva. Además de todos los conocimientos adquiridos, ha sido muy gratificante ver cómo poco a poco la aplicación iba tomando forma y se iba convirtiendo en lo que había imaginado al comienzo del proyecto. Aunque ha sido un proyecto muy ambicioso, he conseguido completarlo con éxito y estoy muy satisfecho con el resultado final.
En cuanto al seguimiento del proyecto, también he logrado completar el proyecto en el tiempo estimado al principio del Sprint. Sin embargo, a veces he tenido que dedicar más tiempo del esperado a la implementación de algunas funcionalidades, como por ejemplo las preguntas de tipo \textit{micrófono} o a las pruebas de integración. 
Considero que mi motivación durante todos los meses de trabajo se ha visto alzada por la temática y el dominio del proyecto, ya que desde pequeño he estado muy interesado en el aprendizaje musical y esta idea la llevaba pensando unos meses antes de comenzar con el proyecto. 


\section{Trabajo futuro}
\label{sec:trabajo_futuro}
Aunque el proyecto ha sido completado con éxito, existen algunas funcionalidades que no han podido ser implementadas ni entraban en el alcance de este proyecto, pero que podrían ser implementadas en un futuro para mejorar la aplicación. Algunas de estas funcionalidades son:
\begin{itemize}
    \item \textbf{Sistema de tienda}: para que los usuarios puedan canjear puntos por premios dentro de la aplicación (insignias, fondos, etc.)
    \item \textbf{Sistema de amigos} que permita a los usuarios añadir a otros usuarios como amigos y ver sus perfiles.
    \item \textbf{Sistema de chat}: Se podría implementar un sistema de chat, que permita a los usuarios comunicarse entre ellos.
    \item \textbf{Sistema de comentarios}: Se podría implementar un sistema de comentarios, que permita a los usuarios comentar las lecciones y las preguntas o incluso sugerir nuevas preguntas para una lección.
    \item \textbf{Sistema de estadísticas} para que los usuarios puedan ver sus estadísticas de progreso en el perfil de forma más detallada.
    \item \textbf{Más tipos de preguntas}, como por ejemplo preguntas de tipo \textit{arrastrar y soltar} o preguntas de tipo \textit{piano} (tocar en un pequeño piano de la pantalla la(s) nota(s) que se pidan).
\end{itemize}

Como toda aplicación, Meloudy continuará evolucionando y mejorando con el tiempo, pero por el momento se puede decir que el proyecto ha sido completado con éxito y que la aplicación está lista para ser utilizada por los usuarios sin ningún tipo de limitación.



   \addcontentsline{toc}{chapter}{10. Bibliografía}
\bibliographystyle{plain}
\bibliography{refs}

  % \addcontentsline{toc}{chapter}{10. Bibliografía}
\bibliographystyle{plain}
\bibliography{refs}

%
%%\chapter{Conclusiones y Trabajos Futuros}
%
%
%%\nocite{*}
%\bibliography{bibliografia/bibliografia}\addcontentsline{toc}{chapter}{Bibliografía}
%\bibliographystyle{miunsrturl}
%
%\appendix
%\input{apendices/manual_usuario/manual_usuario}
%%\input{apendices/paper/paper}
%\input{glosario/entradas_glosario}
% \addcontentsline{toc}{chapter}{Glosario}
% \printglossary
\chapter*{}
\thispagestyle{empty}

\end{document}
