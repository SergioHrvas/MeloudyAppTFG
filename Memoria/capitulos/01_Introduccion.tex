\chapter{Introducción}
La música ha sido siempre una parte importante de la vida de las personas y una pieza clave y fundamental en nuestra sociedad y en nuestras tradiciones.
Escuchamos canciones mientras andamos por la calle, cuando cocinamos, mientras hacemos ejercicio, conduciendo en el coche... La música está presente en gran
parte de nuestra vida cotidiana
y de nuestra cultura y es por esto que, además de ser considerada una de las bellas artes, mucha gente se dedica a estudiarla y a aprender sobre ella.

Sin embargo, por desgracia a lo largo de la historia el aprendizaje de la música ha sido elitista y la
gente con pocos recursos económicos no ha podido acceder a gran parte del conocimiento. En la antigüedad solo
las familias nobles y adineradas podían disfrutar de las melodías clásicas. Puede parecer que no, pero incluso en la actualidad,
aunque mucho menos, la formación musical a veces puede ser vista como un privilegio reservado para unos pocos, pues solo
quienes pueden permitirse ir a una escuela de música o conservatorio pueden llegar a ese conocimiento.

Con la revolución digital y el surgimiento del software, la forma en la que consumimos música ha cambiado radicalmente.
Ahora, gracias a algunas aplicaciones, podemos acceder a una gran cantidad de temas y canciones de forma gratuita y sencilla.
No podemos imaginarnos nuestro día a día sin las aplicaciones que nos ofrecen música en streaming.

Pero, ¿qué pasa con el acceso a el aprendizaje musical? ¿Ha habido realmente un cambio en la forma en la que se estudia y/o practica música
con la aparación de los dispositivos móviles, de internet y de las aplicaciones software? ¿Es posible aprender música de forma autodidacta y
gratuita?


\section{Motivacion}
El acceso a la música es un derecho fundamental de todas las personas y siempre lo he creído así. El surgimiento de nuevas
aplicaciones que ofrecen formas de aprendizaje lúdicas y divertidas de idiomas, matemáticas, geografía, etc me ha hecho pensar
en lo útiles que pueden llegar a ser para personas con pocos recursos que busquen una forma rápida y sencilla de aprender sobre algún tema
en específico fácilmente.

Por otro lado, mi experiencia estudiando en una escuela de música durante 6 años me ha hecho entender que el aprendizaje musical, pese a ser un proceso complejo y exigente,
es gratificante y satisfactorio; y que cualquier persona, tenga o no recursos económicos, debe poder aprender música y disfrutar de ella sin impedimentos y al alcance de su mano.

Al unir el surgimiento de distintas aplicaciones software educativas con mi experiencia en el conservatorio, se me ocurrió
la aparición de un sistema de aprendizaje musical que permita a cualquier persona, independientemente
de su nivel original de conocimientos musicales, aprender conceptos básicos e intermedios sobre lenguaje musical asi como
instruirse para empezar a tocar instrumentos musicales con un enfoque didáctico, divertido y progresivo.



\section{Descripción del problema}



\section{Objetivos}

