\chapter{Introducción}
Desde el inicio de la humanidad, la música ha sido siempre una parte muy importante de la vida de la mayoría de personas y a día de hoy, es una pieza clave
 y fundamental en nuestra sociedad y en nuestras tradiciones. Escuchamos canciones mientras andamos por la calle, 
 cuando cocinamos, mientras hacemos ejercicio, conduciendo en el coche... La música está presente en gran
parte de nuestros hábitos y de nuestra cultura y es por esto que, además de ser considerada una de las
 bellas artes, mucha gente se dedica a estudiarla y a aprender sobre ella.

Sin embargo, por desgracia a lo largo de la historia el aprendizaje del lenguaje musical ha sido elitista y la mayoría
de personas con pocos recursos económicos no ha podido acceder a gran parte de este conocimiento. Antiguamente, solo
las familias nobles y adineradas podían darse el lujo de disfrutar de las melodías clásicas. Puede parecer que no, pero incluso en la actualidad,
aunque mucho menos, la formación musical a veces puede ser vista como un privilegio reservado para unos pocos, pues solo
quienes pueden permitirse ir a una escuela de música o conservatorio pueden llegar a ese aprendizaje.

Con la revolución digital y el surgimiento del software, la forma en la que consumimos música ha cambiado radicalmente.
Ahora, gracias a algunas aplicaciones, podemos acceder a una gran cantidad de temas y canciones de forma gratuita y sencilla.
El impacto ha sido tan fuerte que prácticamente no podemos imaginarnos nuestro día a día sin estas aplicaciones que nos ofrecen música en streaming.

Pero, ¿qué pasa con el acceso a el aprendizaje musical? ¿Ha habido realmente un cambio en la forma en la que se estudia y/o practica música
con la aparación de los dispositivos móviles, de internet y de las aplicaciones software? ¿Es posible aprender música de forma autodidacta y
gratuita?


\section{Motivacion}
El acceso a la música es un derecho fundamental de todas las personas y siempre lo he creído así. El surgimiento de nuevas
aplicaciones que ofrecen formas de aprendizaje lúdicas y divertidas en distintos ámbitos como idiomas, matemáticas, geografía, etc, me ha hecho pensar
en lo útiles que pueden llegar a ser estas para personas con pocos recursos que buscan una forma rápida, barata y sencilla de aprender sobre algún tema
en específico.

Por otro lado, mi experiencia estudiando en una escuela de música durante 6 años me ha hecho entender que el aprendizaje musical, pese a ser un proceso complejo y exigente,
es gratificante y satisfactorio; y que cualquier persona, tenga o no recursos económicos, debería tener la oportunidad de poder aprender música y disfrutar de ella sin impedimentos.
La música y la formación en esta debería ser accesible para cualquier persona.

Al unir el surgimiento de distintas aplicaciones software educativas con mi experiencia en el conservatorio, se me ocurrió
la idea de desarrollar un sistema de aprendizaje musical que permita a cualquier persona, independientemente
de su nivel original de conocimientos musicales, aprender conceptos básicos e intermedios sobre lenguaje musical asi como
instruirse para empezar a tocar instrumentos musicales con un enfoque didáctico, divertido y progresivo.

Todo esto y la falta de aplicaciones similares en el mercado han convertido en una necesidad el desarrollo de este proyecto.

\section{Descripción del problema y objetivos}
Se pretende desarrollar una aplicación software que permita al usuario aprender conceptos de lenguaje musical de forma autodidacta y sencilla,
proporcionando un enfoque que anime a las personas a involucrarse en la adquisición de conocimientos y a entender muchísimos conceptos que la envuelven, desde el lenguaje musical
hasta la práctica de un instrumento. 

También se desea un sistema que proporcione una metodología de enseñanza lúdica y divertida, 
con un progreso organizado en logros y niveles que permita a los usuarios sentirse satisfechos con su progreso y motivados a seguir aprendiendo. 
Los usuarios podrán ver su progreso mediante un sistema de cuentas de usuario que permitirá tambien la posibilidad de compartir sus logros con otros usuarios.

Otro de los objetivos será incorporar distintos tipos de ejercicios y actividades que faciliten este aprendizaje, como por ejemplo la inclusión de un sistema que detecte las notas musicales
tocadas por el usuario. 

Por último, se creará una aplicación web para la gestión de los usuarios y la administración de los contenidos de la aplicación móvil, siendo una 
herramienta más para administradores y personal encargado de la aplicación.

