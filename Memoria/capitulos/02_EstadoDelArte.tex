    \chapter{Estado del Arte}
    \section{Software a desarrollar}
    \subsection{Aplicaciones de aprendizaje}
    Actualmente existen numerosas aplicaciones que ayudan a la adquisición de conocimientos de distintos temas, 
    como los idiomas, las matemáticas o la teoría para la conducción de un vehículo, entre otros. A continuación se
    detallarán las aplicaciones más destacadas.
   
    \subsubsection{Duolingo}
    \begin{wrapfigure}{r}{0.25\textwidth}
        \vspace*{-0.4cm}

        \centering
        \includegraphics[width=0.2\textwidth]{imagenes/c2/duolingo.png}
        \caption{Logo de Duolingo}
        \vspace*{-0.15cm}
    \end{wrapfigure}

    Duolingo es una de las aplicaciones más populares que existen para aprender idiomas. Esta aplicación se basa en el método
    de aprendizaje por inmersión, en el que el usuario estudia el idioma mediante una serie de actividades lúdicas.
    La aplicación se divide en lecciones o temas, las cuales contienen una serie de ejercicios que el usuario debe
    completar para ir avanzando por las distintas secciones. Estos ejercicios son muy variados y consisten en, por ejemplo,
    la traducción de palabras o frases con el teclado, la traducción de estas seleccionando bloques de palabras, la
    pronunciación de palabras o frases mediante el micrófono del dispositivo, la selección de imágenes, etc. Además,
    Duolingo ofrece en cada lección, antes de cada ejercicio, una breve explicación de la gramática o vocabulario con
    imágenes y explicaciones sencillas. Esta aplicación es gratuita (aunque tiene mejoras de pago dentro de esta) 
    y está disponible para dispositivos móviles (Android, iOS y Windows Phone).
 
    \subsubsection{FisicaMaster \& QuímicaMaster}

    \subsubsection{BMath}


    \subsubsection{Academons}

    \subsection{Aplicaciones de aprendizaje musical}
    Centrándonos ya en el tema que nos ocupa, el aprendizaje musical, existen numerosas aplicaciones que ayudan a la adquisición
    de conocimientos musicales o a la ayuda a aprender a tocar un instrumento musical. Algunas de estas aplicaciones son:

    \subsubsection{--}
    \subsubsection{--}
    \subsubsection{--}

    \subsubsection{Guitar Pro}

  



    \section{Desarrollo de Software}
\subsection{Flutter}
\subsection{Kotlin}
\subsection{React native}
\subsection{NodeJS}

\subsection{React}
\subsection{Angular}
\subsection{SQL}