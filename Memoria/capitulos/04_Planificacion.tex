\chapter{Planificacion}
\section{Introducción}
En este capítulo vamos a realizar la planificación para el desarrollo del proyecto, la cual va a servir para poder
realizar un control de los avances del producto y asegurar que se cumplan los objetivos marcados en cada sprint (o iteración).
Para ello, se utilizarán algunos elementos de la metodología SCRUM, como son los sprints, las historias de usuario y el product backlog, etc. Es importante esto último
porque solo usaremos algunos principios y conceptos de metodologías ágiles pero no vamos a seguir al pie de la letra ninguna de ellas debido
a las condiciones en las que se va a desarrollar el proyecto (no hay un cliente real, no hay un equipo de desarrollo real, etc.).

\section{Velocidad}
A continuación vamos a estimar la velocidad inicial de trabajo que se tendrá en el desarrollo. Esta será una aproximación que nos ayudará a estimar la cantidad de trabajo que se debería realizar
en cada iteración. Sin embargo, esta podrá variar y ajustarse a lo largo del proyecto en función de la cantidad de trabajo que se logre completar en cada sprint.

Para calcular la velocidad, vamos a considerar las horas de trabajo que se pretenden dedicar al proyecto cada día y tratar de estimar cuanta cantidad de trabajo se podría realizar.

\begin{itemize}
\item Podríamos decir que partimos de un "equipo de desarrollo" de 1 miembro.
\item Se pretende dedicar 5 horas al día de trabajo aproximadamente.
\item Cada Sprint dura 2 semanas (14 días). Si cada día trabajamos 5 horas aproximadamente, obtenemos un total de 70 horas en cada Sprint, que equivalen a 3 días de trabajo reales.
\item En mi entorno, se estima que 1 PH es un día de trabajo ideal. Esto quiere decir que en cada jornada de trabajo real, se debería completar 1 PH. Sin embargo, en la realidad, esto no siempre es así ya que es una aproximación.
\item Si multiplicamos un programador por los 3 días de trabajo reales, obtenemos que deberíamos completar \textbf{3 PH} por sprint aproximadamente.
\end{itemize}

\section{Product Backlog}
A continuación, se listarán todas las historias de usuario ordenadas por prioridad de acuerdo a la importancia que tienen para el usuario final del producto.
\begin{itemize}
    \item \textbf{HU1 - } Como usuario quiero registrarme en la aplicación para poder comenzar a utilizar sus funcionalidades. (Prioridad: Alta | Puntos de Historia: 1)
    \item \textbf{HU2 - } Como usuario quiero iniciar sesión en la aplicación para poder acceder a sus funcionalidades. (Prioridad: Alta | Puntos de Historia: 1)
    \item \textbf{HU6 - } Como usuario quiero seleccionar una lección para leer el temario. (Prioridad: Alta | Puntos de Historia: 0.5)
    \item \textbf{HU7 - } Como usuario quiero empezar un test de una lección. (Prioridad: Alta | Puntos de Historia: 0.5)
    \item \textbf{HU8 - } Como usuario quiero responder una pregunta de un test del tipo selección múltiple. (Prioridad: Alta | Puntos de Historia: 0.5)
    \item \textbf{HU9 - } Como usuario quiero responder una pregunta de un test del tipo selección única. (Prioridad: Alta | Puntos de Historia: 0.5)
    \item \textbf{HU10 - } Como usuario quiero responder una pregunta de un test del tipo respuesta por micrófono. (Prioridad: Alta | Puntos de Historia: 2)
    \item \textbf{HU11 - } Como usuario quiero ver el resultado de un test. (Prioridad: Media | Puntos de Historia: 0.5)
    \item \textbf{HU13 - } Como usuario quiero ver mi progreso en las lecciones. (Prioridad: Media | Puntos de Historia: 0.5)
    \item \textbf{HU14 - } Como profesor quiero crear una lección. (Prioridad: Media | Puntos de Historia: 0.5)
    \item \textbf{HU15 - } Como profesor quiero modificar el texto de una lección. (Prioridad: Media | Puntos de Historia: 0.5)
    \item \textbf{HU16 - } Como profesor quiero añadir contenido multimedia a una lección. (Prioridad: Media | Puntos de Historia: 1)
    \item \textbf{HU17 - } Como profesor quiero eliminar imágenes de una lección. (Prioridad: Media | Puntos de Historia: 0.5)
    \item \textbf{HU18 - } Como profesor quiero añadir preguntas a un test del tipo selección múltiple. (Prioridad: Media | Puntos de Historia: 0.5)
    \item \textbf{HU19 - } Como profesor quiero añadir preguntas a un test del tipo selección única. (Prioridad: Media | Puntos de Historia: 0.5)
    \item \textbf{HU20 - } Como profesor quiero añadir preguntas a un test del tipo respuesta por micrófono. (Prioridad: Media | Puntos de Historia: 1)
    \item \textbf{HU21 - } Como profesor quiero eliminar preguntas de un test. (Prioridad: Media | Puntos de Historia: 0.25)
    \item \textbf{HU22 - } Como profesor quiero modificar preguntas de un test. (Prioridad: Media | Puntos de Historia: 0.5)
    \item \textbf{HU24 - } Como administrador quiero modificar los datos de un usuario. (Prioridad: Media | Puntos de Historia: 1)
    \item \textbf{HU25 - } Como administrador quiero eliminar un usuario. (Prioridad: Media | Puntos de Historia: 0.5)
    \item \textbf{HU23 - } Como administrador quiero crear un usuario. (Prioridad: Baja | Puntos de Historia: 1)
    \item \textbf{HU12 - } Como usuario quiero ver el ranking de usuarios. (Prioridad: Baja | Puntos de Historia: 0.5)
    \item \textbf{HU5 - } Como usuario quiero ver el perfil de otros usuarios.  (Prioridad: Baja | Puntos de Historia: 0.5)
    \item \textbf{HU3 - } Como usuario quiero ver los datos personales y los logros de mi perfil. (Prioridad: Media | Puntos de Historia: 0.5)
    \item \textbf{HU4 - } Como usuario quiero editar los datos de mi perfil. (Prioridad: Media | Puntos de Historia: 1)


\end{itemize}

\section{Sprints}
Nuestro proyecto se va a desarrollar en distintos Sprints, que son iteraciones de trabajo que se van a realizar en el proyecto.
Cada Sprint tendrá una duración de 2 semanas, y se pretenden realizar
\subsection{Sprint \#1 - Documentación inicial}
\textit{24/11/2022   -   10/12/2022}

En este Sprint se realizará:
\begin{itemize}

    \item Definición del proyecto y del alcance.
    \item Capítulo 1 - Introducción.
    \item Capítulo 2 - Estado del Arte.
\end{itemize}
\subsection{Sprint \#2 - Documentación de requisitos}
En este Sprint se planea realizar:
\textit{10/12/2022   -   12/01/2023}
\begin{itemize}
    \item Revisión de la documentación inicial y corrección de errores.
    \item Capítulo 3 - Especificación de Requisitos.
\end{itemize}

\subsection{Sprint \#3 - Documentación de HU}
\textit{12/01/2023   -   26/02/2023}
En este Sprint se realizará:
\begin{itemize}
    \item Revisión de la documentación de requisitos y corrección de errores.
    \item Capítulo 4 - Planificación.
    \item Capítulo 5 - Análisis del problema.
\end{itemize}
\subsection{Sprint \#4 - Diseño y comienzo del desarrollo}
\textit{26/02/2023   -   09/03/2023}
En este Sprint se terminará la documentación y comenzará el desarrollo con:
\begin{itemize}
    \item Capítulo 6 - Diseño.
    \item Preparación del backend para el desarrollo.
    \begin{itemize}
        \item creación de la base de datos.
        \item creación de los modelos, rutas y controladores del servidor con NodeJS.
    \end{itemize}
    \item Preparación del frontend para el desarrollo (creación de la aplicación de Flutter).
\end{itemize}
\subsection{Sprint \#5 - Registro y login de usuarios}
\textit{09/03/2023   -   23/03/2023}
\begin{itemize}
    \item Realizar la HU1 - Como alumno quiero registrarme en la aplicación.
    \item Realizar la HU2 - Como alumno quiero iniciar sesión en la aplicación.
    \item Realizar la HU6 - Como usuario quiero seleccionar una lección para leer el temario.
    \item Realizar la HU7 - Como usuario quiero empezar un test de una lección. 
\end{itemize}


\subsection{Sprint \#6}
\textit{23/03/2023   -   06/04/2023}
\begin{itemize}
    \item d
    \item d
    \item d
    \item d

\end{itemize}

\subsection{Sprint \#7}
\textit{06/04/2023   -   20/04/2023}
\begin{itemize}
    \item d
    \item d
    \item d
    \item d

\end{itemize}

\subsection{Sprint \#8}
\textit{20/04/2023   -   04/05/2023}


\subsection{Sprint \#9}
\textit{04/05/2023   -   18/05/2023}


\subsection{Sprint \#10}
\textit{18/05/2023   -   01/06/2023}


\subsection{Sprint \#11 - }
\textit{01/06/2023   -   15/06/2023}


\subsection{Sprint \#12 - Pruebas y finalización del proyecto}
\textit{15/06/2023   -   29/06/2023}


\subsection{Sprint \#13 - Pruebas y finalización del proyecto}
\textit{29/06/2023   -   15/07/2023}
\begin{itemize}
    \item Realizar pruebas de integración
    \item Realizar pruebas de unidad
    \item Capítulo 8 - Pruebas
    \item Capítulo 9 - Conclusiones
    \item Revisión de todo el documento y corrección de errores
\end{itemize}



\section{Diagrama de Gantt}