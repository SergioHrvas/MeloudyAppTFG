\chapter{Especificación de requisitos}


\section{Introducción}
A continuación, vamos a describir y detallar los requisitos del sistema 
que vamos a desarrollar. En esta sección se describirá qué vamos a construir y cómo lo vamos
a hacer, con sus restricciones específicas.

\subsection{Propósito}
En este capítulo se pretende describir de forma clara y precisa las funciones, carasterísticas y restricciones 
del sistema que se va a desarrollar. Estas definiciones servirán al equipo de desarrolo para 
conocer las necesidades del sistema y a los usuarios finales. 

Además, este documento servirá como base para el desarrollo de las funcionalidades descritas y como medio de comunicación 
entre las partes involucradas en el proyecto.
\subsection{Ámbito del Sistema}
El sistema que vamos a desarrollar es una aplicación multiplataforma que permitirá a los usuarios
aprender conceptos musicales de una forma amena y divertida.

\subsection{Definiciones, Acrónimos y Abreviaturas}

\begin{itemize}
    \item \textbf{Requisito:} Es una condición o característica que debe cumplir el sistema para satisfacer una necesidad o cumplir una función.
    \item \textbf{Funcionalidad:} Descripción de lo que debe hacer el producto software.
    \item \textbf{Restricción:} Condición que limita la funcionalidad del sistema.
    \item \textbf{Interfaces externas:} 
    \item \textbf{Rendimiento: }
    \item \textbf{Usuarios/Cliente: }
    \item \textbf{}
\end{itemize}

\subsection{Referencias}
\subsection{Visión General del Documento}


\section{Descripción General}
\subsection{Perspectiva del producto}
\subsection{Funciones del producto}
\subsection{Características de los usuarios}
\subsection{Restricciones}
\subsection{Suposiciones y dependencias}


\section{Requisitos Específicos}
\subsection{Requisitos funcionales}
\subsection{Requisitos no funcionales}
\subsection{Requisitos de información}
\subsection{Bocetos de interfaz de usuario}