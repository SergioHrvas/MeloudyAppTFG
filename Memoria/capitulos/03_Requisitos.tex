\chapter{Especificación de requisitos}


\section{Introducción}
A continuación, vamos a describir y detallar los requisitos del sistema 
que vamos a desarrollar. En esta sección se describirá qué vamos a construir y cómo lo vamos
a hacer, con sus restricciones específicas.

\subsection{Propósito}
En este capítulo se pretende describir de forma clara y precisa las funciones, carasterísticas y restricciones 
del sistema que se va a desarrollar. Estas definiciones servirán al equipo de desarrolo para 
conocer las necesidades del sistema y a los usuarios finales. 

Además, este documento servirá como base para el desarrollo de las funcionalidades descritas y como medio de comunicación 
entre las partes involucradas en el proyecto.
\subsection{Ámbito del Sistema}
El sistema que vamos a desarrollar es una aplicación multiplataforma llamada "Meloudy" que permitirá a los usuarios
aprender conceptos musicales de una forma amena y divertida. La aplicación llevará el progreso de los usuarios que les permitirá saber cómo van en su aprendizaje.

Las gestiones que se realizarán en el sistema son:
\begin{itemize}
    \item \textbf{Gestión de usuarios: } El sistema permitirá a los administradores crear y eliminar usuarios, además de modificar sus datos. Los usuarios también podrán
    modificar sus datos.
    \item \textbf{Gestión de lecciones: } El sistema facilitará la modificación de los textos de las lecciones y de las preguntas asociadas a estas.
\end{itemize}


\subsection{Definiciones, Acrónimos y Abreviaturas}
A continuación se detallará el significado de algunos conceptos importantes para la comprensión del documento y de nuestro sistema.
\begin{itemize}
    \item \textbf{Requisito:} Es una condición o característica que debe cumplir el sistema para satisfacer una necesidad o cumplir una función.
    \item \textbf{Funcionalidad:} Descripción de lo que debe hacer el producto software.
    \item \textbf{Restricción:} Condición que limita la funcionalidad del sistema.
    \item \textbf{Interfaces externas:} Tipo de requisito que incluye la interfaz de usuario (interacción entre el software y el usuario), los diseños de pantallas, las interfaces hardware y software, etc.
    \item \textbf{Usuario: }Persona que utilizará la aplicación para la intención final de esta.
    \item \textbf{Administrador: } Persona encargada del sistema software y del mantenimiento de este para su buen funcionamiento.
    \item \textbf{RF / RNF:} Requisito funcional / Requisito No Funcional.
\end{itemize}

\subsection{Referencias}
Esta sección de Especificación de requisitos ha sido redactada consultando los documentos del estándar IEEE Recommended Practice for Software Requirements Specification ANSI/IEEE 830, 1998.

\subsection{Visión General del Documento}
El documento consta de tres partes bien diferenciadas:
\begin{itemize}
    \item \textbf{Introducción:} Proporciona una visión general sobre el apartado de Especificación de Requisitos sin profundizar en los requisitos como tal.
    \item \textbf{Descripción General:} Se describirá el sistema a construir para saber las funciones principales, los datos necesarios, las restricciones y otros aspectos que puedan afectar al desarrollo de la aplicación.
    \item \textbf{Requisitos Específicos:} Se profundiza en las necesidades del usuario definiendo los requisitos que debe tener nuestro sistema tras el desarrollo y la implementación de este.
\end{itemize}

\section{Descripción General}
A continuación, se procederá a describir con poco detalle los requisitos del sistema de una forma general para saber las funciones principales a desarrollar, las características de los usuarios, las dependencias, etc.
\subsection{Perspectiva del producto}
Se pretende implementar un sistema que permita el aprendizaje del usuario mediante técnicas de gamificación y ejercicios a resolver. Algunas de estas actividades utilizarán librerías para la detección de notas musicales
a partir de la frecuencia captada por el micrófono del usuario y, por tanto, la aplicación dependerá de dichas librerías que se incluyan.
Por otro lado, el sistema de administración y el progreso de los usuarios se llevará a cabo utilizando una base de datos en la que se almacenará la información necesaria de los usuarios y su progreso.

La interacción de los usuarios con la aplicación será mediante una interfaz gráfica que podrá ser utilizada con la pantalla táctil o el ratón del dispositivo usado.

\subsection{Funciones del producto}
Las principales funciones que el usuario podrá realizar dentro de la aplicación son:
\begin{itemize}
    \item Selección de lecciones a aprender.
    \item Respuesta (mediante selección o escritura) de las preguntas y actividades que ofrezca la aplicación.
    \item Consulta del progreso individual, de los logros obtenidos y de la clasifiación global.
    \item Modificación de los datos del perfil.
\end {itemize}

Además, el encargado/profesor se encargará de parte de la administración de las lecciones mediante:
\begin{itemize}
    \item Modificación y creación del texto de la lección
    \item Modificación de las preguntas y actividades
    \item Creación de preguntas para una lección
\end{itemize}

\subsection{Características de los usuarios}
Los usuarios que usarán la aplicación tendrán distintos perfiles y abarcarán edades muy distintas. Aún así,
el perfil objetivo para el uso del sistema será las personas jóvenes, pues suelen utilizar mucho más las nuevas tecnologías
y estarán más familiarizados con este tipo de aplicaciones. No se necesita conocimiento musical previo para utilizar el software y, 
de hecho, las lecciones pueden ser bastante básicas y sencillas para las personas que ya conozcan conceptos y aspectos avanzados del lenguaje musical.


\subsection{Restricciones}
\subsection{Suposiciones y dependencias}


\section{Requisitos Específicos}
\subsection{Requisitos funcionales}
\subsection{Requisitos no funcionales}
\subsection{Requisitos de información}
\subsection{Bocetos de interfaz de usuario}