\chapter{Pruebas}
\label{cap:pruebas}
En este capítulo se describen las pruebas realizadas al sistema, tanto de forma individual como de forma integrada. Se describen los casos de prueba, los resultados obtenidos y las conclusiones extraídas de los mismos.
El objetivo de las pruebas es comprobar que el sistema cumple con los requisitos establecidos en el capítulo \ref{cap:especificacion-requisitos} y que funcione correctamente.

\section{Paquetes utilizados}
\label{sec:paquetes-utilizados}
Para realizar las pruebas se han utilizado los siguientes paquetes:

\begin{itemize}
    \item \textbf{flutter\_test}: paquete que contiene las clases necesarias para realizar las pruebas de unidad.
    \item \textbf{flutter\_driver}: paquete que contiene las clases necesarias para realizar las pruebas de integración.
\end{itemize}

\section{Pruebas de unidad}
\label{sec:pruebas-unidad}
Las pruebas de unidad son aquellas que se realizan sobre los componentes más pequeños del sistema, como pueden ser las funciones o los métodos. Estas pruebas se realizan de forma individual, aislando el componente a probar de los demás componentes del sistema. 
De esta forma nos aseguramos de que el componente funciona correctamente y que no depende de otros componentes para su correcto funcionamiento.

\subsection{Lecciones}
\label{subsec:pruebas-unidad-lecciones}




\section*{Pruebas de controlador o widget}
\label{sec:pruebas-controlador}
Las pruebas de controlador o widget son aquellas que se realizan sobre los controladores o widgets de la aplicación. Estas pruebas se realizan de forma individual, aislando el controlador o widget a probar de los demás componentes del sistema.


\section*{Pruebas de sistema o de integración}
\label{sec:pruebas-sistema}
Las pruebas de sistema o de integración son aquellas que se realizan sobre el sistema completo. Estas pruebas se realizan de forma conjunta, comprobando que todos los componentes del sistema funcionan correctamente y que se comunican entre ellos de forma correcta.
De esta forma nos aseguramos de que el sistema funciona de forma adecuada simulando varias acciones que podría realizar un usuario.
